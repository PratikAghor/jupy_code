\documentclass{article}
\usepackage[utf8]{inputenc}
\usepackage{amsmath}
\usepackage{float}
\usepackage{listings}
\usepackage[pdftex]{graphicx}
\usepackage{amssymb}
\usepackage{subcaption}
%--------------------------------------------------------------------

\author{Pratik Aghor}
\title{SRI Blog}
\date{\today}  % Toggle commenting to test

\begin{document}

\maketitle
%--------------------------------------------------------------------
\section*{SRI: Linear Stability}
%--------------------------------------------------------------------
%--------------------------------------------------------------------
\subsection*{Introduction}
%--------------------------------------------------------------------

We start the analysis with trying reproduce results of \cite{robins2020viscous}. The physical problem is that of stratified Taylor-Couette flow. 
TODO: Define dimensionless parameters, describe the problem, write full nonlinear-governing equations and derive the linearized equations (keep nonlinear terms till third order - for nonlinear stability later).

\begin{equation}\label{def:Re}
 Re = \frac{r_{i}\Omega_{i}d}{\nu}
\end{equation}

The base state is $(0, \Omega r, 0)$, with 
 \noindent
\begin{equation}
\Omega = A + \frac{B}{r^{2}},
\qquad
A = \frac{\mu - \eta^{2}}{1-\eta^{2}},
\qquad
B = \frac{\eta^{2}(1-\mu)}{(1+\eta)(1-\eta)^{3}},
\qquad
Z = 2A,
\label{eqn:base_state}
\end{equation}

Here $Z = (1/r) \partial_{r}(r^{2}\Omega)$ is the constant base-state vorticity. 
Centrifugal approximation $r\Omega^{2}<<g$ is made to make sure that the base state density stratification is only in the $z$ direction and the constant base-state density contours do not have curvature in the radial direction. The bouyancy frequency $N^{2} = -(g/\rho_{0})d\rho_{0}/dz$ is also assumed to be constant. This can be achieved by making the base state density $\rho_{0}$ a linear function of $z$ so $d\rho_{0}/dz$ is constant. Also, under the Boussinesq approximation, $\rho_{0}$ is assumed more or less constant, but for a weak variation in the axial direction in the bouyancy terms. The Froude number is defined as 
\begin{equation}\label{def:Fr}
 Fr = \Omega_{i}/N
\end{equation}
such that low Froude number would corrospond to strong stratification. Linear instability modes of the base state are assumed to be of the form  $\tilde{A}(r) \exp{(\sigma_{c}t + i k z + i m \theta)}$, where $\tilde{A}(r)$ is the $r$-dependent complex amplitude. 


%--------------------------------------------------------------------
\subsection*{Linearized Equations:}
%--------------------------------------------------------------------
With these assumptions the linearized equations become:
\begin{subequations}
 \begin{align}
  \begin{split}
  -i\Phi u_{r} & - 2\Omega u_{\theta} \\
  &= -\frac{dP}{dr} + \frac{\eta}{(1-\eta)} \frac{1}{Re}\bigg[\frac{d^{2}u_{r}}{dr^{2}} + \frac{1}{r}\frac{du_{r}}{dr} - \bigg( \frac{m^{2} + 1}{r^{2}} + k^{2}\bigg)u_{r} - \frac{2 i m u_{\theta}}{r^{2}} \bigg],
  \end{split}
 \end{align}
 %
 \begin{align}
  \begin{split}
   -i\Phi u_{\theta} & + Z u_{r}\\
   & = \frac{-imP}{r} + \frac{\eta}{(1-\eta)} \frac{1}{Re}\bigg[\frac{d^{2}u_{\theta}}{dr^{2}} + \frac{1}{r}\frac{du_{\theta}}{dr} - \bigg( \frac{m^{2} + 1}{r^{2}} + k^{2}\bigg)u_{\theta} + \frac{2 i m u_{r}}{r^{2}}\bigg],
  \end{split}
 \end{align}
 %
 \begin{align}
 -i\Phi u_{z} = -ik P - \rho  + \frac{\eta}{(1-\eta)} \frac{1}{Re}\bigg[\frac{d^{2}u_{z}}{dr^{2}} + \frac{1}{r}\frac{du_{z}}{dr}- \bigg( \frac{m^{2} + 1}{r^{2}} + k^{2}\bigg)u_{z} \bigg],
 \end{align}
 %
 \begin{align}
  -i \Phi \rho - Fr^{-2}u_{z} = 0,
 \end{align}
 %
 \begin{align}
  \frac{du_{r}}{dr} + \frac{u_{r}}{r} + \frac{i m u_{\theta}}{r} + i k u_{z} = 0.
 \end{align}
\end{subequations}
Here $(u_{r}, u_{\theta}, u_{z}, \rho, P)$ are complex amplitudes of perturbations in the radial, azimutha, axial velocities and in density and pressure fields respectively. The $\Phi(r) = u\sigma_{c} - m \Omega(r) = i\sigma + \omega - m \Omega(r)$ is called the Lagrangian frequency. 

The no slip boundary conditions for the viscous flow are defined as 
\begin{equation}
 u_{r} = u_{\theta} = u_{z} = 0 
 \qquad
 \textrm{at } r = r_{i} = \frac{\eta}{1-\eta} \textrm{ and } r = r_{o} = \frac{1}{1-\eta} 
\end{equation}

%--------------------------------------------------------------------

%--------------------------------------------------------------------

%--------------------------------------------------------------------
%--------------------------------------------------------------------
\bibliographystyle{apalike}
%\bibliographystyle{unsrt} % Use for unsorted references  
%\bibliographystyle{plainnat} % use this to have URLs listed in References
%\cleardoublepage
%\bibliography{References/references} % Path to your References.bib file

\bibliography{bib/references} % Path to your References.bib file
 \if@openright\cleardoublepage\else\clearpage\fi
 \cleardoublepage
 \pagestyle{empty}
%--------------------------------------------------------------------
\end{document}
