\documentclass{jfm}

\usepackage{graphicx}
%\usepackage{epstopdf,epsfig}
\usepackage{newtxtext}
\usepackage[varvw]{newtxmath}
\usepackage{natbib}
\usepackage{hyperref}
\hypersetup{
    colorlinks = true,
    urlcolor   = blue,
    citecolor  = black,
}
\newtheorem{lemma}{Lemma}
\newtheorem{corollary}{Corollary}
\newcommand{\RomanNumeralCaps}[1]
\linenumbers
% {\MakeUppercase{\romannumeral #1}}

\title{Stratorotational Instability}

\author{Pratik Aghor\aff{1}
  \corresp{\email{pratik.aghor54@gmail.com}},
  }

\affiliation{\aff{1} Department of Mathematics and Statistics, University of New Hampshire, USA}

\begin{document}
\maketitle

\begin{abstract}
We study stratorotational instability (SRI) with centrifugal bouyancy. 
\end{abstract}

\begin{keywords}
SRI, Taylor-Couette, stratified flows, instability, bifurcations, nonlinear dynamics, pattern formation
\end{keywords}

% {\bf MSC Codes }  {\it(Optional)} Please enter your MSC Codes here
%--------------------------------------------------------------------
\section{Introduction}
\label{sec:intro}
%--------------------------------------------------------------------

\noindent We start out with trying to reproduce the linear stability results of \citet{lopez_marques_avila_2013} using deadalus's eigenvalue solvers (eigtools). Specifically, the goal is to include centrifugal bouyancy in the Boussinesq approximation and compare the critical Grashof number ($G$) where the instability sets in as a function of the inner Reynolds number $Re_{i}$. 


We use the governing equations derived in \cite{lopez_marques_avila_2013}. 


\begin{subeqnarray}\label{eq:gov_eqns}
  (\partial_{t} + \boldsymbol{v \cdot\nabla})\boldsymbol{v} & = & - \nabla p + \nabla^{2}\boldsymbol{v} + \alpha g T \hat{z} + \alpha T {\boldsymbol{v}\cdot\nabla}\boldsymbol{v},\\[3pt]
  (\partial_{t} + \boldsymbol{v\cdot\nabla})T & = & \kappa \nabla^{2} T,\\[3pt]
  \boldsymbol{\nabla \cdot v} & = & 0.
\end{subeqnarray}

As noted in \cite{lopez_marques_avila_2013}, the traditional Boussinesq approximation can be recovered by replacing the centrifugal term $\alpha T {\boldsymbol{v}\cdot\nabla}\boldsymbol{v}$ by $-\epsilon T {\boldsymbol{v_{b}}\cdot\nabla}\boldsymbol{v_{b}}$ where $v_{b}$ is the known Couette-flow base velocity, given in \cite{chandrasekhar2013hydrodynamic} and can be expressed as $v_{b}(r) = Ar + B/r$. See \cite{lopez_marques_avila_2013} for details.

The physical system is that of a standard Taylor-Couette setup with periodic boundaries in both $\theta$- and $z$-directions. The inner and outer cylinders rotate independently at angular velocities $\Omega_{i}$ and $\Omega_{o}$ respectively.\cite{lopez_marques_avila_2013} study quasi-Keplerian regime ($\Omega_{i} > \Omega_{o}$, $r_{i}^{2}\Omega_{i}< r_{o}^{2}\Omega_{o}$ ). A negative temperature gradient in the radial direction is considered with the outer wall temperature fixed at $T_{c} + \Delta T/ 2 $ and inner wall held at $T_{c} - \Delta T/ 2 $, where $T_{c}$ is the mean temperature.

The dimensionless equations use the gap width $d = r_{o} -r_{i}$ as the length scale, viscous time $d^{2}/\nu$ as the time scale, $\Delta T$ as the temperature scale and $(d/\nu)^{2}$ as the pressure scale. Moreover, $T$ is the deviation from the mean temperature $T_{c}$. 

Accordingly, the velocity and temperature boundary conditions become:
\begin{subeqnarray}\label{eq:bc}
  u = w = 0 \textrm{ at } r & = &  r_{i}, r_{o},\\[3pt]
  v = Re_{i} \textrm{ at } r = r_{i} & \textrm{, } & v = Re_{o} \textrm{ at } r = r_{o},\\[3pt]
  T = -1/2 \textrm{ at } r =  r_{i} & \textrm{, } & T =  1/2 \textrm{ at } r = r_{o},
\end{subeqnarray}

where $r_{i} = \frac{\eta}{1 - \eta}$ and $r_{o} = \frac{1}{1 - \eta}$.
%--------------------------------------------------------------------
\subsection{Base Flow}
%--------------------------------------------------------------------
As given in \citet{lopez_marques_avila_2013}, we use the standard Couette flow profile for azimuthal velocity and impose the no-mass-flux condition axially to obtain an axial base flow that fixes the pressure field. The base flow for temperature is that corresponding to a conduction profile. The axial no-mass-flux condition can be written as:
\begin{equation}\label{eq:axial_no_mass_flux}
 \int_{r_{i}}^{r_{o}} r w_{b} dr = 0
\end{equation}

The steady base-flow is then given by:
\begin{subeqnarray}\label{eq:base_flow}
  u_{b}(r) & = & 0 \\[3pt]
  v_{b}(r) & = & Ar + B/r \\[3pt]
  w_{b}(r) & = & G\bigg(C(r^{2} - r^{2}_{i}) + (C(r^{2}_{o} - r^{2}_{i}) + \frac{1}{4}(r^{2}_{o} - r^{2}))\frac{\ln{(r/r_i)}}{\ln{\eta}} \bigg) \\[3pt]
  T_{b}(r) & = & \frac{1}{2} + \frac{\ln{(r/r_i)}}{\ln{\eta}} \\[3pt]
  p(r, z) & = & p_{0} + G\bigg( 4C + \frac{1}{2} - \frac{1}{\ln{\eta}} \bigg) z + \int_{r_{i}}^{r} (1- \epsilon T_{b} r) v_{b}^{2} \frac{dr}{r}  
\end{subeqnarray}
%--------------------------------------------------------------------
\subsection{Perturbed Equations:}
%--------------------------------------------------------------------
We perturb the governing equations Eqns. \ref{eq:gov_eqns} about the base-flow and substitute $\phi = \phi_{b} + \phi$ where $\phi_{b}$ are as given in Eqns. \ref{eq:base_flow} and $\phi$'s on now become the purbturbations. 

\begin{subeqnarray}\label{eq:perturbed_eqns}
   \partial_{t}v &=& -(\boldsymbol{v_{b} \cdot\nabla}\boldsymbol{v_{b}} + \boldsymbol{v_{b} \cdot\nabla}\boldsymbol{v} + \boldsymbol{v \cdot\nabla}\boldsymbol{v_{b}} + \boldsymbol{v \cdot\nabla}\boldsymbol{v})\nonumber\\
  && -\nabla p_{b} - \nabla p + \nabla^{2}\boldsymbol{v_{b}} + \nabla^{2}\boldsymbol{v} + \alpha g T \hat{z} \nonumber\\
  && + \alpha T {\boldsymbol{v}\cdot\nabla}\boldsymbol{v},   \\[3pt]
  (\partial_{t} + \boldsymbol{v\cdot\nabla})T & = & \kappa \nabla^{2} T,\\[3pt]
  \boldsymbol{\nabla \cdot v} & = & 0.
\end{subeqnarray}


%--------------------------------------------------------------------

%\bibliographystyle{jfm}
%\bibliography{jfm}
%Use of the above commands will create a bibliography using the .bib file. Shown below is a bibliography built from individual items. 

\bibliographystyle{jfm}
\bibliography{bib/jfm} % Path to your References.bib file
 \if@openright\cleardoublepage\else\clearpage\fi
 \cleardoublepage
 \pagestyle{empty}
%--------------------------------------------------------------------

\end{document}
