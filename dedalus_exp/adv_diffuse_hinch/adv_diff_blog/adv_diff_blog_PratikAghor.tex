\documentclass{article}
\usepackage[utf8]{inputenc}
\usepackage{amsmath}
\usepackage{float}
\usepackage{listings}
\usepackage[pdftex]{graphicx}
\usepackage{amssymb}
\usepackage{subcaption}
\usepackage{newtxtext}
\usepackage[varvw]{newtxmath}

%--------------------------------------------------------------------

\newcommand{\del}[1]{\partial{#1}}
\newcommand{\delsq}[1]{\partial^{2}{#1}}
\newcommand{\BCs}{boundary conditions}
%---------------------------------------------------------------
\title{Asymptotics of Advection Diffusion - Blog}

\author{Pratik Aghor}

% \affiliation{\aff{1} Department of Mathematics and Statistics, University of New 
% Hampshire, USA}
%---------------------------------------------------------------

\title{}
\date{\today}  % Toggle commenting to test

\begin{document}

\maketitle
%--------------------------------------------------------------------
\section*{Advection Diffusion}
%--------------------------------------------------------------------
We take a general advection diffusion equation from Sec. $7.3$ of \cite{hinch1991perturbation}
\begin{equation}\label{eq:gov_eqn}
   \frac{\del{f}}{\del{t}} + \frac{\del{}}{\del{x}} (\omega(\theta)f)= \epsilon \frac{\delsq{f}}{\del{\theta^{2}}}
\end{equation}

Where $\theta \in [0, 2\pi)$ with periodic \BCs in $\theta$. We also assume $\omega(\theta) > 0$ in the domain and $O(1)$ for all times. A good example would be $\omega(\theta) = 2 + \sin{\theta}$. 

There are two physical processes happening, advection on the fast time-scale $\tau = t$ and diffusion on the slow time-scale$T = \epsilon t$.\cite{hinch1991perturbation} has done the standard multiple scales analysis, which we will repeat here. On top of the two time-scales, spatially, we have a small-parameter attached to the diffusion term. The question we are trying to answer here is the following: Are there any internal/boundary layers due to the singular nature of the spatial problem? We also intend to do the multiple-scales analysis in conjuction to the matched-asymptotics expansion. 
%--------------------------------------------------------------------
\subsection{The Method of Multiple Scales:}
We pose the expansion:
\begin{equation}\label{eq:mult_scales_exp}
 f(\theta, t; \epsilon) = f_{0}(\theta, \tau, T) + \epsilon f_{1}(\theta, \tau, T) + \ldots
\end{equation}
Substituting Eqn.(\ref{eq:mult_scales_exp}) into Eqn.(\ref{eq:gov_eqn}) and collecting terms at different orders, we obtain:
\begin{subequations}
 \begin{align*}
  O(\epsilon^{0}) &: \frac{\del{f_{0}}}{\del{\tau}} + \frac{\del{(\omega f_{0})}}{\del{\theta} } = 0\\
  \textrm{i.e.}\\
  &\frac{1}{\omega}\bigg[\frac{\del{}}{\del{\tau}} + \omega \frac{\del{}}{\del{\theta}}  \bigg] f_{0} = 0
 \end{align*}
\end{subequations}

%--------------------------------------------------------------------
%--------------------------------------------------------------------
\bibliographystyle{apalike}
%\bibliographystyle{unsrt} % Use for unsorted references  
%\bibliographystyle{plainnat} % use this to have URLs listed in References
%\cleardoublepage
%\bibliography{References/references} % Path to your References.bib file

\bibliography{bib/references} % Path to your References.bib file
 \if@openright\cleardoublepage\else\clearpage\fi
 \cleardoublepage
 \pagestyle{empty}
%--------------------------------------------------------------------
\end{document}
